\documentclass[10pt]{article}
\usepackage[margin=1in]{geometry}
\usepackage[numbers]{natbib}
\usepackage{alltt}
\usepackage{color}
\usepackage{booktabs}
\usepackage{url}

\newcommand{\PAR}[1]{\par\smallskip\noindent\textbf{#1}}
\newcommand{\ra}[1]{\renewcommand{\arraystretch}{#1}}

\begin{document}

% Emotions. What are the benefits of a computer program understanding
% emotions? How about a computer having emotions? Propose how to do
% this. See GBDM pages 95-97, 104.

\title{The VOTE Program as Open Source}
\author{George V. Neville-Neil}
\maketitle
%
\section{Introduction}
\label{sec:intro}

The \verb|VOTE| system is a Common Lisp implementation of the work
presented in the book Goal Based Decision Making by Stephen Slade
\url{https://www.amazon.com/Goal-based-Decision-Making-Interpersonal-Model/dp/0805813667}.
The code has evolved over several decades and is now available on
github as a git repository. \url{https://github.com/sslade/VOTE}.  At
the time of this writing the system only contains information on the
legislative bills, issues, and people of the United States in the late
1980s.  Replacing the databases used by the system would allow it to
be used in other political systems.

An example of how to run the current system is given in the
\verb|README.md| in the root of the repository.

The rest of this document contains a brief overview of the code, for
those who wish to extend it, as well as a discussion of how this code
can be maintained as an ongoing, open source, project.

\section{Code Overview}
\label{sec:overview}

All of the code for the system is contained under the \verb|lisp|
directory in the root of the repository.  There remain some lingering
vestiges of past code but we shall ignore those for now and only
concentrate on the actively used components.

The \verb|pol/| directory contains the main pieces of code that handle
the goal based reasoning described in the book listed in
Section~\ref{sec:intro}.  
\begin{table*}
  \centering
  \begin{tabular}{|l|l|}
\hline
    File & Purpose \\
    \hline
    bill.lisp & Legislative bills, past and present\\ 
    group.lisp & Special Interest Groups \\
    issue.lisp & Societal and other issues that might go into a bill\\
    member.lisp & Members of the legislature \\
    stance.lisp  & Handling the stances of members on issues\\
\hline
  \end{tabular}
  \caption{Major Code Files}
  \label{tab:code}
\end{table*}

Each of the major sections has its own lisp file, shown in
Table~\ref{tab:code}.  The code is written in an object oriented (OO)
style where each major component is given a class, and each class has
defined methods.  For example, the \verb|issue| class is defined in
the \verb|issue.lisp| file, and contains all the code that is
necessary to interact with the issues database as well as respond to
queries sent to it.  All of the major code sections follow this same
pattern.

The language generator, which currently only handles English, is
contained int he \verb|gen/| directory.  It has a relatively good
implementation of English grammar, suitable to the task making the
output of issues, stances, bills and votes less taxing to the reader.
Some amount of randomness is injected into the system so that each
time a query is made of the system the output might be a bit
different.  The \verb|gen| code has a test entry point that is useful
in seeing how this works.

The data for the system is contained in Lisp textual databases found
under the \verb|db/data/| directory.  Whilst these databases can be
manipulated directly there are also commands in each class
(\verb|bill|, \verb|issue| etc.) that can be used to update the
database entries.

\section{Ideas for Future Work}
\label{sec:future}

For the VOTE program to be a viable open source project it needs to
have more engagement from people who find the concepts interesting,
and want to apply them to other areas of goal based reasoning.  The
system is generic enough to allow for expansion in several different
ways.

\subsection{Language Translation}
\label{sec:lang}

One of the suggested projects for students who worked with the system
in the course at Yale, where this code originated, was to translate
the system into another language.  In fact there had been a French
translation in earlier code but this was lost to the winds of time,
and only a few vestiges remained.  The code in the \verb|gen/|
directory is very much focused on English grammar and English
political turns of phrase.  Nonetheless it was possible to do a
translation into Japanese, although without any updates to the
political arena as data on Japanese bills would have been hard to come
by.  A fork of the code with Japanese was submitted as a project in
2024.

The current language code is not table driven, and the text and code
are mixed together into each of the relevant classes
(e.g. \verb|noun.lisp|, \verb|verb.lisp| etc.) A fully
internationalized system would have to switch to tables for words and
phrases, and different classes for each language, as the grammar is
encoded in the language class, e.g. \verb|english.lisp|,
\verb|japanese.lisp| etc.

A more full translation effort should be tied to creating a version of
the databases so that they inhabit a different political arena as
discussed next.

\subsection{New Political Arenas}
\label{sec:arenas}

The databases associated with the code only contain people, groups,
issues and bills that relate to the US House of Representatives as of
the late 1980s.  The format of the data is clear and so importing
other sets of data is straightforward, but getting access to that data
may be a challenge.  Political systems exist within independent
language and cultural systems, which means that even trying to bringing
databases that are nominally in English, for example from the UK
Parliament, will require some amount of linguistic changes, as
discussed in Section~\ref{sec:lang}.  Anyone who has read or heard
political discussions in both the US House and UK Parliament knows
that the language used, while nominally English, differs in important
respects that change the meaning of words.  Anything further afield
from the US and the UK would, obviously, require an entirely new
language translation of the \verb|gen/| code.

\subsection{New Areas of Reasoning}
\label{sec:reasoning}

The \verb|VOTE| system is a specific instance of goal based reasoning,
one which handles roll call voting.  The main base classes are capable
of being adapted to other areas of goal based reasoning, by being
abstracted away from the specific words used (issue, stance, bill
etc.).  A much longer term project would be to do this abstraction so
that other types of systems could be tested against this type of reasoning.

\section{Conclusion}

The \verb|VOTE| system is now available as an open source project.
With the accompanying book it can be updated and extended under a
clear open source license and used for further experiments in goal
based reasoning.  We have provided three example ideas for how these
experiments might proceed, but we expect others in the open source
community to give the code a try and see what they can make of it.

\footnotesize{
\bibliographystyle{acm}
\bibliography{main}
}

\end{document}
\endinput

%%% Local Variables:
%%% mode: LaTeX
%%% TeX-master: t
%%% End:
